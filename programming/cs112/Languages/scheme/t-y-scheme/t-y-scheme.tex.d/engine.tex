\chapter{Engines}

\index{engine}
An engine \cite{engine} represents computation that is
subject to timed preemption.  In other words, an
engine's underlying computation is an ordinary thunk
that runs as a timer-preemptable process.

An engine is called with three arguments: (1) a number
of time units or {\em ticks}, (2) a {\em success}
procedure, and (3) a {\em failure} procedure.  If the
engine computation finishes within the allotted {\em
ticks}, the {\em success} procedure is applied to the
computation result and the remaining ticks.  If the
engine computation could not finish within the allotted
{\em ticks}, the {\em failure} procedure is applied
to a new engine representing the unfinished portion of
the engine computation.

For example, consider an engine whose underlying
computation is a loop that printed the nonnegative
integers in sequence.  It is created as follows, with
the soon-to-be-defined \q{make-engine} procedure.
\q{make-engine} is called on an argument thunk
representing the underlying computation, and it
returns the corresponding engine:

\q{
(define printn-engine
  (make-engine
    (lambda ()
      (let loop ((i 0))
        (display i)
        (display " ")
        (loop (+ i 1))))))
}

\n Here is a call to \q{printn-engine}:

\q{
(define *more* #f)
(printn-engine 50 list (lambda (ne) (set! *more* ne)))
|evalsto 0 1 2 3 4 5 6 7 8 9
}

\n Ie, the loop gets to print upto a certain number
(here \q{9}) and then fails because of the clock
interrupt.  However, our {\em failure} procedure sets
a global variable called \q{*more*} to the failed
engine, which we can use to resume the loop where the
previous engine left off:

\q{
(*more* 50 list (lambda (ne) (set! *more* ne)))
|evalsto 10 11 12 13 14 15 16 17 18 19
}

\index{call/cc@\q{call/cc}!and engine}

We will now construct engines using \q{call/cc} to
capture the unfinished computation of a failing engine.
First we will construct {\em flat} engines, or
engines whose computation cannot include the running of
other engines.  We will later generalize the code to
the more general {\em nestable} engines or {\em
nesters}, which can call other engines.  But in both
cases, we need a timer mechanism, or a {\em clock}.

\index{clock}

\section{The clock}
\label{engine-clock}

Our engines assume the presence of a global clock or
interruptable timer that marks the passage of ticks as
a program executes.  We will assume the following clock
interface --- if your Scheme provides any kind of alarm
mechanism, it should be an easy matter to rig up a
clock of the following type. (Appendix \ref{clock}
defines a clock for the Guile \cite{guile} dialect of
Scheme.)

The internal state of our \q{clock} procedure consists
of two items:

(1) the number of remaining ticks; and

(2) an interrupt handler to be invoked when the
clock runs out of ticks.

\q{clock} allows the following operations:

(1) \q{(clock 'set-handler h)} sets the
interrupt handler to \q{h}.

(2) \q{(clock 'set n)} resets the clock's
remaining ticks to \q{n}, returning the
previous value.

The number of ticks ranges over the non-negative
integers and an atom called \q{*infinity*}.  A clock with
\q{*infinity*} ticks cannot run out of time and so will
not set off the interrupt handler.  Such a clock is
{\em quiescent} or ``already stopped''.  To stop a
clock, set its ticks to \q{*infinity*}.

The clock handler is set to a thunk.  For example,

\q{
(clock 'set-handler
  (lambda ()
    (error "Say goodnight, cat!")))

(clock 'set 9)
}

\n This will cause an error to be signaled after 9
ticks have passed, and the message displayed by the
signal will be ``Say goodnight, cat!''


\index{engine!flat}
\section{Flat engines}

We will first set the clock interrupt handler.  Note
that the handler is invoked only if a non-quiescent
clock runs out of ticks.  This happens only during
engine computations that are on the brink of failure,
for only engines set the clock.

The handler captures the current continuation, which is
the rest of the computation of the currently failing
engine.  This continuation is sent to another
continuation stored in the global \q{*engine-escape*}.
The \q{*engine-escape*} variable stores the exit
continuation of the current engine.  Thus the clock
handler captures the rest of the failing engine and
sends it to an exit point in the engine code, so the
requisite failure action can be taken.

\scmfilename engine.scm

\scmdribble{
(define *engine-escape* #f)
(define *engine-entrance* #f)

(clock 'set-handler
  (lambda ()
    (call/cc *engine-escape*)))
}        

Let us now look into the innards of the engine code
itself.  As said, \q{make-engine} takes a thunk and
fashions an engine out of it:

\scmdribble{
(define make-engine
  (lambda (th)
    (lambda (ticks success failure)
      (let* ((ticks-left 0)
             (engine-succeeded? #f)
             (result
              (call/cc
               (lambda (k)
                 (set! *engine-escape* k)
                 (let ((result
                        (call/cc
                         (lambda (k)
                           (set! *engine-entrance* k)
                           (clock 'set ticks)
                           (let ((v (th)))
                             (*engine-entrance* v))))))
                   (set! ticks-left (clock 'set *infinity*))
                   (set! engine-succeeded? #t)
                   result)))))
        (if engine-succeeded?
            (success result ticks-left)
            (failure 
             (make-engine 
              (lambda () 
                (result 'resume)))))))))
}

\n First we introduce the variables \q{ticks-left}
and \q{engine-succeeded?}.  The first will hold
the ticks left over should the engine thunk finish
in time.  The second is a flag that will be used in
the engine code to signal if the engine suceeded.

We then run the engine thunk within two nested calls to
\q{call/cc}.  The first \q{call/cc} captures the
continuation to be used by a failing engine to abort
out of its engine computation.  This continuation is
stored in the global \q{*engine-escape*}.  The second
\q{call/cc} captures an inner continuation that 
will be used by the return value of the thunk \q{th} if
it runs to completion.  This continuation is stored
in the global
\q{*engine-entrance*}.

Running through the code, we find that after capturing
the continuations \q{*engine-escape*} and
\q{*engine-entrance*}, we set the clock's ticks to the
time allotted this engine and run the thunk \q{th}.  If
\q{th} succeeds, its value \q{v} is sent to the
continuation \q{*engine-entrance*}, after which the
clock is stopped, the remaining ticks ascertained, and
the flag \q{engine-succeeded?} is set to true.  We now
go past the \q{*engine-escape*} continuation, and run
the final dispatcher in the code: Since we know the
engine succeeded, we apply the \q{success} procedure to
the result and the ticks left.

If the thunk \q{th} {\em didn't} finish in time
though, it will suffer an interrupt.  This invokes the
clock interrupt handler, which captures the current
continuation of the running and now failing thunk and
sends it to the continuation \q{*engine-escape*}.  This
puts the failed-thunk continuation in the outer
\q{result} variable, and we are now in the final
dispatcher in the code: Since \q{engine-succeeded?} is
still false, we apply the \q{failure} procedure to new
engine fashioned out of \q{result}.

Notice that when a failed engine is removed, it will
traverse the control path charted by the first run of
the original engine.  Nevertheless, because we have
explicitly use the continuations stored in the global
variables \q{*engine-entrance*} and
\q{*engine-escape*}, and we always set them anew before
executing an engine computation, we are assured that
the jumps will always come back to the currently
executing engine code.

\index{engine!nestable}
\section{Nestable engines}

In order to generalize the code above to accommodate
the nestable type of engine, we need to incorporate
into it some {\em tick management} that will take
care of the apportioning of the right amounts of ticks
all the engines in a nested run.

To run a new engine (the {\em child}), we need to
stop the currently engine (the {\em parent}).  We
then need to assign an appropriate number of ticks to
the child.  This may not be the same as the ticks
assigned by the program text, because it would be {\em
unfair} for a child to consume more ticks than its
parent has left.  After the child completes, we need to
update the parent's ticks.  If the child finished in
time, any leftover ticks it has revert to the parent.
If ticks were denied from the child because the parent
couldn't afford it, then if the child fails, the parent
will fail too, but must remember to restart the child
with its promised ticks when it (the parent) restarts.

We also need to \q{fluid-let} the globals
\q{*engine-escape*} and \q{*engine-entrance*}, because
each nested engine must have its own pair of these
sentinel continuations.  As an engine exits (whether
through success or failure), the \q{fluid-let} will
ensure that the next enclosing engine's sentinels take
over.

Combining all this, the code for nestable engines looks
as follows:

\scmfilename nestable-engine.scm
\scmdribble{
(define make-engine
  (lambda (th)
    (lambda (ticks s f)
      (let* ((parent-ticks 
              (clock 'set *infinity*))

             ;A child can't have more ticks than its parent's
             ;remaining ticks
             (child-available-ticks 
              (clock-min parent-ticks ticks))

             ;A child's ticks must be counted against the parent
             ;too
             (parent-ticks-left
              (clock-minus parent-ticks child-available-ticks))

             ;If child was promised more ticks than parent could
             ;afford, remember how much it was short-changed by
             (child-ticks-left 
              (clock-minus ticks child-available-ticks))

             ;Used below to store ticks left in clock
             ;if child completes in time
             (ticks-left 0)

             (engine-succeeded? #f)

             (result
              (fluid-let ((*engine-escape* #f)
                          (*engine-entrance* #f))
                (call/cc
                 (lambda (k)
                   (set! *engine-escape* k)
                   (let ((result
                          (call/cc
                           (lambda (k)
                             (set! *engine-entrance* k)
                             (clock 'set child-available-ticks)

                             (let ((v (th)))

                               (*engine-entrance* v))))))
                     (set! ticks-left
                       (let ((n (clock 'set *infinity*)))
                         (if (eqv? n *infinity*) 0 n)))
                     (set! engine-succeeded? #t)
                     result))))))

        ;Parent can reclaim ticks that child didn't need
        (set! parent-ticks-left
          (clock-plus parent-ticks-left ticks-left))

        ;This is the true ticks that child has left --
        ;we include the ticks it was short-changed by
        (set! ticks-left 
          (clock-plus child-ticks-left ticks-left))

        ;Restart parent with its remaining ticks
        (clock 'set parent-ticks-left)
        ;The rest is now parent computation

        (cond
         ;Child finished in time -- celebrate its success
         (engine-succeeded? (s result ticks-left))

         ;Child failed because it ran out of promised time --
         ;call failure procedure
         ((= ticks-left 0)
          (f (make-engine (lambda () (result 'resume)))))

         ;Child failed because parent didn't have enough time,
         ;ie, parent failed too.  If so, when parent is
         ;resumed, its first order of duty is to resume the
         ;child with its fair amount of ticks
         (else
          ((make-engine (lambda () (result 'resume)))
           ticks-left s f)))))))
}

Note that we have used the arithmetic operators
\q{clock-min}, \q{clock-minus}, and \q{clock-plus}
instead of \q{min}, \q{-}, and \q{+}.  This is because
the values used by the clock arithmetic includes
\q{*infinity*} in addition to the integers.  Some Scheme
dialects provide an \q{*infinity*} value in their
arithmetic\f{Eg, in Guile, you can \q{(define
*infinity* (/ 1 0))}.} --- if so, you can use the regular
arithmetic operators.  If not, it is an easy exercise
to define the enhanced operators.





