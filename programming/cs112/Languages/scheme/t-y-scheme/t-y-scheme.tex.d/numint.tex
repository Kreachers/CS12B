\chapter{Numerical techniques}
\label{numint}

\index{numerical integration}
\index{Simpson's rule}
Recursion (including iteration) combines well with
Scheme's mathematical primitive procedures to implement
various numerical techniques.  As an example,
let's implement Simpson's rule, a procedure for finding
an approximation for a definite integral.  

\section{Simpson's rule}

The definite integral of a function $f(x)$ within an
interval of integration $[a,b]$ can be viewed as the
{\em area under the curve} representing $f(x)$ from the
lower limit $x = a$ to the upper limit $x = b$.
In other words, we consider the graph of the curve for
$f(x)$ on the $x,y$-plane, and find the area enclosed
between that curve, the $x$-axis, and the {\em
ordinates} of $f(x)$ at $x = a$ and $x = b$.  

\verbwritefile numint.mp
\verbwrite{
% File: t-y-scheme.mp
beginfig(1);

u = 1mm;

x.max = 120u;
y.max = 70u;

% draw x- and y-axes
drawarrow (-5u,0) -- (x.max,0);
drawarrow (0,-5u) -- (0,y.max);

% label the axes
label.rt(btex $x$ etex, (x.max,0) shifted (5,0));
label.top(btex $y$ etex, (0,y.max) shifted (0,5));

x0 = 25u;

% interval between x[i] and x[i+1] is h
h = 5u;

% number of intervals (minus 1)
n = 14;

% calculate all the x[i]
for i = 1 upto n:
  x[i] = x0 + i*h;
%  drawdot(x[i],0);
endfor;

% specify a function f(x) that is defined 
% for all x in [ x0, x[n] ]
path f;
f = (x0-5u,40u) .. (x0+2h,55u) .. (x0+4h,65u) ..
(x0+6h,55u) .. (x0+7h,55u) .. (x[n]+10u,50u); 
draw f;     

% label f(x)
label.rt(btex $f(x)$ etex, point infinity of f);

% x[i] is a good-looking abcissa in the middle
% of [ x0, x[n] ] 
%i = 8;
i = n div 2;

path ord[];

% calculate (and draw) the ordinates for f(x)
% at x0, x[i], x[i+1], x[n]
for j = 0,i,i+1,n:
z[j]f = ( (x[j],0)--(x[j],y.max) ) intersectionpoint f;
ord[j] = (x[j],0) -- (x[j],y[j]f);
draw ord[j];
endfor;

%% gray the integral area
%path integ;
%integ = buildcycle(((x0-5,0)--(x[n]+5,0)),
%          ( (x0,-1) .. ord0 .. (x0,y0f+1)), f, 
%((x[n],-1) ..  ord[n] .. (x[n], y[n]f+1))) ;
%fill integ withcolor .8white;

% identify x[i]--x[i+1] distance as h
path hline;
hline = (x[i],2h)--(x[i+1],2h);

drawdblarrow hline;
label.top(btex $h$ etex, point .5 of hline);

% label the abcissae x0, x[i], x[i+1], x[n]

dotlabel.bot(btex $\strut x_0 = a$ etex, (x0,0));
dotlabel.bot(btex $\strut x_i$ etex, (x[i],0));
dotlabel.lrt(btex $\strut x_{i+1}$ etex, (x[i+1],0));
dotlabel.bot(btex $\strut x_n = b$ etex, (x[n],0));

% label the corresponding ordinates 
label.lft(btex $y_0$ etex, point .5 of ord0);
label.lft(btex $y_i$ etex, point .5 of ord[i]);
label.rt(btex $y_{i+1}$ etex, point .5 of ord[i+1]);
label.rt(btex $y_n$ etex, point .5 of ord[n]); 


endfig;
end.
}
\centerline{\epsfbox{numint.1}}

According to Simpson's rule, we divide 
the interval of integration $[a,b]$ 
into $n$ evenly spaced
intervals, where $n$ is even. (The larger $n$
is, the better the approximation.)
The interval boundaries constitute
$n+1$ points on the $x$-axis, viz, $x_0$, $x_1$,
\dots, $x_i$, $x_{i+1}$, \dots, $x_n$, where
$x_0 = a$ and $x_n = b$.
The length of each
interval is $h = (b - a)/n$, so each $x_i = a+ih$.   We then
calculate the ordinates of $f(x)$ at the interval
boundaries.  There are $n+1$ such ordinates, viz,
$y_0$, \dots, $y_i$, \dots, $y_n$, where  $y_i = f(x_i)
= f(a+ih)$.
Simpson's rule
approximates the definite integral of $f(x)$ between
$a$ and $b$ with the value\f{Consult any elementary
text on the calculus for an explanation of why this
approximation is reasonable.}:

$$
{h\over 3} \left[ (y_0 + y_n) 
+ 4(y_1 + y_3 + \cdots + y_{n-1}) 
+ 2(y_2 + y_4 + \cdots + y_{n-2}) \right]
$$
%
We define
the procedure \q{integrate-simpson}
to take four arguments: the integrand \q{f}; the
$x$-values at the limits \q{a} and \q{b}; and the
number of intervals \q{n}.  

\scmfilename simpson.scm
\scmdribble{
(define integrate-simpson
  (lambda (f a b n)
    ;...
}

\n The first thing we do in
\q{integrate-simpson}'s body is ensure that
\q{n} is even --- if it isn't, we simply bump its
value by 1.  

\scmdribble{
    ;...
    (unless (even? n) (set! n (+ n 1)))
    ;...
}

\n Next, we put in the local variable \q{h}
the length of the interval.  We introduce two more local variables
\q{h*2} and \q{n/2} to store the values of twice \q{h}
and half \q{n} respectively, as we expect to
use these values often in the ensuing calculations.

\scmdribble{
    ;...
    (let* ((h (/ (- b a) n))
           (h*2 (* h 2))
           (n/2 (/ n 2))
           ;...
}

\n We note that the sums $y_1 + y_3 + \cdots + y_{n-1}$
and $y_2 + y_4 + \cdots + y_{n-2}$ both involve adding
every other ordinate. So let's define a local procedure
\q{sum-every-other-ordinate-starting-from} that
captures this common iteration.  By abstracting this
iteration into a procedure, we avoid having to repeat
the iteration textually.  This not only reduces
clutter, but reduces the chance of error, since we have
only one textual occurrence of the iteration to debug.

\q{sum-every-other-ordinate-starting-from} takes two arguments:
the starting ordinate and the number of ordinates to be summed.

\scmdribble{
           ;...
           (sum-every-other-ordinate-starting-from
             (lambda (x0 num-ordinates)
               (let loop ((x x0) (i 0) (r 0))
                 (if (>= i num-ordinates) r
                     (loop (+ x h*2)
                           (+ i 1)
                           (+ r (f x)))))))
           ;...
}         

\n We can now calculate the three ordinate sums, and
combine them to produce the final answer.  Note
that there are $n/2$ terms in $y_1 + y_3 + \cdots +
y_{n-1}$, and $(n/2) - 1$ terms in $y_2 + y_4 + \cdots
+ y_{n-2}$.

\scmdribble{
           ;...
           (y0+yn (+ (f a) (f b)))
           (y1+y3+...+y.n-1
             (sum-every-other-ordinate-starting-from 
               (+ a h) n/2))
           (y2+y4+...+y.n-2
             (sum-every-other-ordinate-starting-from 
               (+ a h*2) (- n/2 1))))
      (* 1/3 h
         (+ y0+yn
            (* 4.0 y1+y3+...+y.n-1)
            (* 2.0 y2+y4+...+y.n-2))))))
}

\n Let's use \q{integrate-simpson} to find the definite
integral of the function 

$$
\phi(x) = {1\over \sqrt{2\pi}} e^{-x^2/2}
$$
%
We first define $\phi$ in Scheme's prefix
notation.\f{$\phi$
is the probability density of a
random variable with a {\em normal} or {\em Gaussian}
distribution, with mean = 0 and standard deviation = 1.
The definite integral $\int_0^z \phi(x) dx$
is the probability that the random
variable assumes a value between 0 and $z$.  
However, you don't need to know all this in
order to understand the example!}

\scmdribble{
(define *pi* (* 4 (atan 1)))

(define phi
  (lambda (x)
    (* (/ 1 (sqrt (* 2 *pi*)))
       (exp (- (* 1/2 (* x x)))))))
}

\n Note that we exploit the fact that $\tan^{-1} 1 =
\pi/4$ in order to define \q{*pi*}.\f{If Scheme didn't
have the \q{atan} procedure, we could use our
numerical-integration procedure to get an approximation
for $\int_0^1 (1 + x^2)^{-1} dx$, which is $\pi/4$.}
 
The following calls calculate the definite integrals of
\q{phi} from 0 to 1, 2, and 3 respectively.  They
all use 10 intervals.

\q{
(integrate-simpson phi 0 1 10)
(integrate-simpson phi 0 2 10)
(integrate-simpson phi 0 3 10)
}

\n To four decimal places, these values should be
0.3413, 0.4772, and 0.4987 respectively \cite[Table
26.1]{hmf}.  Check to see that our implementation
of Simpson's rule does indeed produce comparable
values!\f{By pulling constant factors --- such as \q{(/
1 (sqrt (* 2 *pi*)))} in \q{phi} --- out of the
integrand, we could speed up the ordinate calculations
within \q{integrate-simpson}.} 

\section{Adaptive interval sizes}

It is not always convenient to specify the number \q{n}
of intervals.  A number that is good enough for
one integrand may be woefully inadequate for another.  In
such cases, it is better to specify the amount of
{\it tolerance\/} \q{e} we are willing to grant the final answer, and let
the program figure out how many intervals are needed.  A
typical way to accomplish this is to have the program
try increasingly better answers by steadily increasing
\q{n}, and stop when two successive sums differ within
\q{e}.  Thus:

\q{
(define integrate-adaptive-simpson-first-try
  (lambda (f a b e)
    (let loop ((n 4) 
               (iprev (integrate-simpson f a b 2))) 
      (let ((icurr (integrate-simpson f a b n)))
        (if (<= (abs (- icurr iprev)) e)
            icurr
            (loop (+ n 2)))))))
}

\n Here we calculate successive Simpson integrals (using
our original procedure \q{integrate-simpson}) for
\q{n} = 2, 4, \dots.  (Remember that \q{n} must be even.)
When the integral \q{icurr} for the current \q{n}
differs within \q{e} from the integral \q{iprev} for
the immediately preceding \q{n}, we return \q{icurr}.
                                               
One problem with this approach is that we don't take
into account that only some {\it segments\/} of the
function benefit from the addition of intervals.  For
the other segments, the addition of intervals merely
increases the computation without contributing to a
better overall answer.  For an improved adaptation, we
could split the integral into adjacent segments, and
improve each segment separately.

\q{
(define integrate-adaptive-simpson-second-try
  (lambda (f a b e)
    (let integrate-segment ((a a) (b b) (e e))
      (let ((i2 (integrate-simpson f a b 2))
            (i4 (integrate-simpson f a b 4)))
        (if (<= (abs (- i2 i4)) e)
            i4
            (let ((c (/ (+ a b) 2))
                  (e (/ e 2)))
              (+ (integrate-segment a c e)
                 (integrate-segment c b e))))))))
}

\n The initial segment is from \q{a} to \q{b}.  To find
the integral for a segment, we calculate the Simpson
integrals \q{i2} and \q{i4} with the two smallest
interval numbers 2 and 4.  If these are within \q{e} of
each other, we return \q{i4}.  If not we split the
segment in half, recursively calculate the integral
separately for each segment, and add.  In
general, different segments at the same level converge
at their own pace.  Note that when we integrate a half
of a segment, we take care to also halve the tolerance,
so that the precision of the eventual sum does not
decay.

There are still some inefficiencies in this
procedure: The integral \q{i4} recalculates three
ordinates already determined by \q{i2}, and the integral
of each half-segment recalculates three ordinates 
already determined by \q{i2} and \q{i4}.  We avoid
these inefficiencies by making explicit the sums used
for \q{i2} and \q{i4}, and by transmitting more parameters
in the named-\q{let} \q{integrate-segment}.  This makes for
more sharing, both within the body of \q{integrate-segment}
and across successive calls to \q{integrate-segment}:

\scmdribble{
(define integrate-adaptive-simpson
  (lambda (f a b e)
    (let* ((h (/ (- b a) 4))
           (mid.a.b (+ a (* 2 h))))
      (let integrate-segment ((x0 a)
                              (x2 mid.a.b)
                              (x4 b)
                              (y0 (f a))
                              (y2 (f mid.a.b))
                              (y4 (f b))
                              (h h)
                              (e e))
        (let* ((x1 (+ x0 h))
               (x3 (+ x2 h))
               (y1 (f x1))
               (y3 (f x3))
               (i2 (* 2/3 h (+ y0 y4 (* 4.0 y2))))
               (i4 (* 1/3 h (+ y0 y4 (* 4.0 (+ y1 y3)) 
                               (* 2.0 y2)))))
          (if (<= (abs (- i2 i4)) e)
              i4
              (let ((h (/ h 2)) (e (/ e 2)))
                (+ (integrate-segment
                     x0 x1 x2 y0 y1 y2 h e)
                   (integrate-segment
                     x2 x3 x4 y2 y3 y4 h e)))))))))
}

\n \q{integrate-segment} now explicitly sets four
intervals of size \q{h}, giving five ordinates \q{y0},
\q{y1}, \q{y2}, \q{y3}, and \q{y4}.  The integral
\q{i4} uses all of these ordinates, while the integral
\q{i2} uses just \q{y0}, \q{y2}, and \q{y4}, with an
interval size of twice \q{h}.  It is easy to verify that
the explicit sums used for \q{i2} and \q{i4} do correspond
to Simpson sums.
 
Compare the following approximations of
$\int_0^{20} e^x dx$:

\q{
(integrate-simpson          exp 0 20 10)
(integrate-simpson          exp 0 20 20)
(integrate-simpson          exp 0 20 40)
(integrate-adaptive-simpson exp 0 20 .001)
(- (exp 20) 1)
}

\n The last one is the analytically correct answer.  See
if you can figure out the smallest \q{n} (overshooting is
expensive!)
such that \q{(integrate-simpson exp 0 20 n)} yields a result
comparable to that returned by the \q{integrate-adaptive-simpson}
call.

\input gamma
