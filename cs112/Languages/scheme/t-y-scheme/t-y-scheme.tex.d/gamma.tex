\section{Improper integrals}

Simpson's rule cannot be directly applied to {\em
improper integrals} (integrals such that either
the value of the integrand is unbounded somewhere
within the interval of integration, or the interval of
integration is itself unbounded).  However, the rule can still be 
applied for a {\em part} of the integral, with the
remaining being approximated by other means.  For
example, consider the $\Gamma$ function.
For $n > 0$, $\Gamma(n)$ is defined as the
following integral with unbounded upper limit:

$$ \Gamma(n) = \int_0^{\infty} x^{n-1}e^{-x} dx $$
%
From this, it follows that (a) $\Gamma(1) = 1$, and (b)
for $n > 0$, $\Gamma(n+1) = n\Gamma(n)$.  This implies
that if we know the value of $\Gamma$ in the interval
$(1, 2)$, we can find $\Gamma(n)$ for any {\em real} $n > 0$.
Indeed, if we relax the condition $n > 0$, we can use
result (b) to extend the domain of $\Gamma(n)$ to
include $n \le 0$, with the understanding that the function
will diverge for {\em integer} $n \le 0$.\f{$\Gamma(n)$ for real $n > 0$ is
itself an extension of the ``decrement-then-factorial'' function
that maps {\em integer} $n > 0$ to $(n-1)!$.}

We first implement a Scheme procedure \q{gamma-1-to-2}
that requires its argument \q{n} to be within the
interval $(1, 2)$.  \q{gamma-1-to-2} takes a
second argument \q{e} for the tolerance.

\scmdribble{
(define gamma-1-to-2
  (lambda (n e)
    (unless (< 1 n 2)
      (error 'gamma-1-to-2 "argument outside (1, 2)"))
    ;...
}

\n
We introduce a local variable \q{gamma-integrand} to hold
the $\Gamma$-integrand $g(x) = x^{n-1}e^x$:

\scmdribble{
    ;...
    (let ((gamma-integrand
            (let ((n-1 (- n 1)))
              (lambda (x)
                (* (expt x n-1)
                   (exp (- x))))))
          ;...
}
             
\n We now need to integrate $g(x)$ from 0 to $\infty$.
Clearly we cannot deal with an infinite number of
intervals; we therefore use Simpson's rule for only a
portion of the interval $[0, \infty)$, say $[0, x_c]$
($c$ for ``cut-off'').  For the remaining, ``tail'',
interval $[x_c, \infty)$, we use a tail-integrand
$t(x)$ that reasonably approximates $g(x)$, but has the
advantage of being more tractable to analytic solution.
Indeed, it is easy to see that for sufficiently large
$x_c$, we can replace $g(x)$ by an exponential decay
function $t(x) = y_c e^{-(x - x_c )}$, where $y_c
= g(x_c)$.  Thus:

$$
\int_0^{\infty} g(x) dx \approx 
\int_0^{x_c} g(x) dx +
\int_{x_c}^{\infty} t(x) dx 
$$
%
The first integral can be solved using Simpson's rule,
and the second integral is just $y_c$.  To find $x_c$,
we start with a low-ball value (say 4), and then refine
it by successively doubling it until the ordinate at
$2x_c$ (ie, $g(2x_c)$) is within a certain
tolerance of the ordinate predicted by the tail-integrand 
(ie, $t(2x_c)$).  For both the Simpson
integral and the tail-integrand calculation, we will
require a tolerance of \q{e/100}, an order of 2 less than
the given tolerance \q{e}, so the overall
tolerance is not affected:

\scmdribble{
          ;...
          (e/100 (/ e 100)))
       (let loop ((xc 4) (yc (gamma-integrand 4)))
         (let* ((tail-integrand
                  (lambda (x)
                    (* yc (exp (- (- x xc))))))
                (x1 (* 2 xc))
                (y1 (gamma-integrand x1))
                (y1-estimated (tail-integrand x1)))
           (if (<= (abs (- y1 y1-estimated)) e/100)
               (+ (integrate-adaptive-simpson 
                    gamma-integrand
                    0 xc e/100)
                  yc)
               (loop x1 y1)))))))
}

\n We can now write a more general procedure \q{gamma}
that returns $\Gamma(n)$ for any real $n$:

\scmdribble{
(define gamma
  (lambda (n e)
    (cond ((< n 1) (/ (gamma (+ n 1) e) n))
          ((= n 1) 1)
          ((< 1 n 2) (gamma-1-to-2 n e))
          (else (let ((n-1 (- n 1)))
                  (* n-1 (gamma n-1 e)))))))
}

\n
Let us now calculate $\Gamma(3/2)$.

\q{
(gamma 3/2 .001)
(* 1/2 (sqrt *pi*))
}

\n The second value is the analytically correct answer.
(This is because $\Gamma(3/2) =
(1/2)\Gamma(1/2)$, and $\Gamma(1/2)$ is known to be
\ifx\shipout\UNDEFINED
$\pi^{1/2}$\else
$\sqrt{\pi}$\fi.)
You can modify \q{gamma}'s  second
argument (the tolerance) to get as close an
approximation as you desire.




